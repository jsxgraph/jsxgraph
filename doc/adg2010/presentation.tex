\documentclass{beamer}

\def\xcolorversion{2.00}
\def\xkeyvalversion{1.8}

\usepackage[version=0.96]{pgf}
\usepackage{tikz}
\usetikzlibrary{arrows,shapes,snakes,automata,backgrounds,petri}

\usepackage{hyperref}
\usepackage{listings}

\title{Automatic calculation of plane loci using Groebner bases and integration into a Dynamic Geometry System}   
\author{Michael Gerh\"auser, Alfred Wassermann} 
\date{July 24, 2010} 

\newcommand{\GEONEXT}{GEONE\kern-.06em \lower.5ex\hbox{x}\kern-.215em T}
\definecolor{GXT}{rgb}{0,0.549019608,0}
\setbeamercolor{title}{fg=GXT}
\setbeamercolor{frametitle}{fg=GXT}
\setbeamercolor{blocktitle}{fg=GXT}
\setbeamercolor{section in toc}{fg=GXT}
\usebackgroundtemplate{\includegraphics[width=\paperwidth]{img/background.png}}
\logo{\includegraphics[height=0.75cm]{img/ubt.png}}
\beamertemplatenavigationsymbolsempty

\begin{document}

\frame{\titlepage} 

\frame{\frametitle{Overview}\tableofcontents}

\section{JSXGraph - A short overview} 

% JSXGraph

\frame{
  \frametitle{}

  \begin{block}{JSXGraph}
  \end{block}
}

\frame{
  \frametitle{JSXGraph}

  \begin{block}{What is JSXGraph?}
    \begin{itemize}
      \item A library implemented in JavaScript
      \item Runs in recent versions of all major browsers
      \item No plugins required
      \item LGPL-Licensed
    \end{itemize}
  \end{block}

  \begin{block}{Main features}
    \begin{itemize}
      \item Dynamic Geometry
      \item Interactive function plotting
      \item Turtle Graphics
      \item Charts
    \end{itemize}
  \end{block}
}

\frame{
  \frametitle{JSXGraph} 
  \begin{block}{Supported Hardware}
    \begin{itemize}
      \item PC (Windows, Linux, Mac)
      \item "Touchpads" like the Apple iPad
      \item Mobile phones, iPod
      \item Basically every device which runs at least one of the supported browsers
    \end{itemize}
  \end{block}
}

\frame{
  \frametitle{JSXGraph} 
  \begin{block}{Supported Browsers}
    \begin{itemize}
      \item Firefox
      \item Chrome/Chromium
      \item Safari
      \item Internet Explorer
      \item Opera
    \end{itemize}
  \end{block}
}

\lstset{ %
language=Java,                % choose the language of the code
basicstyle=\tiny,       % the size of the fonts that are used for the code
numbers=none,                   % where to put the line-numbers
numberstyle=\footnotesize,      % the size of the fonts that are used for the line-numbers
stepnumber=1,                   % the step between two line-numbers. If it's 1 each line 
                                % will be numbered
numbersep=5pt,                  % how far the line-numbers are from the code
backgroundcolor=\color{white},  % choose the background color. You must add \usepackage{color}
showspaces=false,               % show spaces adding particular underscores
showstringspaces=false,         % underline spaces within strings
showtabs=false,                 % show tabs within strings adding particular underscores
frame=single,	                % adds a frame around the code
tabsize=2,	                % sets default tabsize to 2 spaces
captionpos=b,                   % sets the caption-position to bottom
breaklines=true,                % sets automatic line breaking
breakatwhitespace=false,        % sets if automatic breaks should only happen at whitespace
title=\lstname,                 % show the filename of files included with \lstinputlisting;
                                % also try caption instead of title
escapeinside={\%*}{*)},         % if you want to add a comment within your code
morekeywords={*,...},           % if you want to add more keywords to the set
morestring=[d][\color{blue}]{'},
morecomment=[s][\color{gray}]{<}{>},
morecomment=[s][\itshape\color{gray}]{[.}{.]}
}

\begin{frame}[fragile]
  \frametitle{JSXGraph} 
  \begin{block}{Example/Input}
    \begin{lstlisting}
<link rel="stylesheet" type="text/css" href="css/jsxgraph.css" />
<script type="text/javascript" src="js/jsxgraphcore.js"></script>
...
<div id="jxgbox" class="jxgbox" style="width:500px; height:500px;"></div>
<script type="text/javascript">
  board = JXG.JSXGraph.initBoard('jxgbox', {boundingbox: [-2, 20, 20, -2], axis: true, grid: false, keepaspectratio: true});
  A = board.create('point', [8, 3]);
  B = board.create('point', [8, 8]);
  c1 = board.create('circle', [B, 4]);
  D = board.create('glider', [0, 0, c1], {name: 'D'});
  g = board.create('line', [A, D]);
  c2 = board.create('circle', [D, 3]);
  T = board.create('intersection', [c2,g,0], {name: 'T'});
</script>
    \end{lstlisting}
  \end{block}
\end{frame}

\frame{
  \frametitle{JSXGraph}

  \begin{block}{Example/Output}
\begin{center}
      \href{http://localhost/~michael/jxg/examples/adg/limacon.html}{\includegraphics[width=5cm]{img/jsx-example.png}}
\end{center}
  \end{block}
}


\frame{
  \frametitle{JSXGraph}
  
  \begin{block}{Supported file formats}
    \begin{itemize}
      \item \GEONEXT{}
      \item GeoGebra
      \item Intergeo
      \item Cinderella (small feature subset)
    \end{itemize}
  \end{block}
}

\begin{frame}[fragile]
  \frametitle{JSXGraph} 
  \begin{block}{Example/Input}
    \begin{lstlisting}
<link rel="stylesheet" type="text/css" href="css/jsxgraph.css" />
<script type="text/javascript" src="js/jsxgraphcore.js"></script>
...
<div id="jxgbox" class="jxgbox" style="width:500px; height:500px;"></div>
<script type="text/javascript">
  board = JXG.JSXGraph.loadBoardFromFile('jxgbox', 'watt.xml', 'intergeo');
  
  function computeLocus() {
    board.create('locus', [JXG.getRef('T')]);
  }
</script>
    \end{lstlisting}
  \end{block}
\end{frame}

\frame{
  \frametitle{JSXGraph}

  \begin{block}{Example/Output}
\begin{center}
      \href{http://localhost/~michael/jxg/examples/adg/limacon.html}{\includegraphics[width=5cm]{img/jsx-example.png}}
\end{center}
  \end{block}
}

% Groebner bases stuff

\section{Computing plane loci using Groebner bases}

\frame{
  \frametitle{}

  \begin{block}{Computing plane loci using Groebner bases (in a nutshell)}
  \end{block}
}

\frame{
  \frametitle{Computing plane loci using Groebner bases}

  \begin{block}{}
    \begin{itemize}
      \item Given a set of free and dependent points,
\begin{center}
      \href{http://localhost/~michael/jxg/examples/adg/limacon.html}{\includegraphics[width=5cm]{img/limacon-base.png}}
\end{center}
    \end{itemize} 
  \end{block}
}

\frame{
  \frametitle{Computing plane loci using Groebner bases}

  \begin{block}{}
    \begin{itemize}
      \item we first choose a coordinate system,
\begin{center}
      \href{http://localhost/~michael/jxg/examples/adg/limacon.html}{\includegraphics[width=5cm]{img/limacon-coords.png}}
\end{center}
    \end{itemize} 
  \end{block}
}

\frame{
  \frametitle{Computing plane loci using Groebner bases}

  \begin{block}{}
    \begin{itemize}
      \item translate geometric constraints into an algebraic form,
\begin{itemize}
  \item $(u[1]-8)^2 + (u[2]-8)^2 - 16 = 0$
  \item $(x-u[1])^2 + (y-u[2])^2 - 9 = 0$
  \item $3x-3u[1]+yu[1]-8y+8u[2]-xu[2] = 0$
\end{itemize}
    \end{itemize} 
  \end{block}
}

\frame{
  \frametitle{Computing plane loci using Groebner bases}

  \begin{block}{}
    \begin{itemize}
      \item calculate the Gr\"obner basis of the given ideal,
\begin{itemize}
  \item $x^6+3x^4y^2+3x^2y^4+y^6-48x^5-38x^4y-96x^3y^2-76x^2y^3-48xy^4-38y^5+1047x^4+1216x^3y+1774x^2y^2+1216xy^3+727y^4-13024x^3-16596x^2y-16096xy^2-8404y^3+97395x^2+109888xy+63535y^2-415536x-300806y+790009 = 0$
\end{itemize}
    \end{itemize} 
  \end{block}
}

\frame{
  \frametitle{Computing plane loci using Groebner bases}

  \begin{block}{}
    \begin{itemize}
      \item and finally plot the calculated implicit equation.
\begin{center}
      \href{http://localhost/~michael/jxg/examples/adg/limacon.html}{\includegraphics[width=5cm]{img/limacon-locus.png}}
\end{center}
    \end{itemize} 
  \end{block}
}

\section{Implementing this algorithm in JSXGraph}

\frame{
  \frametitle{}

  \begin{block}{Implementing this algorithm in JSXGraph}
  \end{block}
}

\frame{
  \frametitle{Implementation}

  \begin{block}{Problems}
    \begin{itemize}
      \item No JavaScript implementation of any Gr\"obner basis algorithm
      \item Can't use C-libraries directly in JavaScript
      \item No implicit plotting in JSXGraph by now
    \end{itemize}
  \end{block}
}

\frame{
  \frametitle{Implementation}

  \begin{block}{AJAX}
    \begin{itemize}
      \item Transfer data (a)synchronously via HTTP with JavaScript
    \end{itemize}
  \end{block}

  \begin{block}{This enables us to}
    \begin{itemize}
      \item use a computer algebra system on a (web) server for the expensive Gr\"obner basis calculations
      \item use a plotting tool/library for implicit plotting
    \end{itemize}
  \end{block}
}

\frame{
  \frametitle{Implementation}

\begin{center}
      \includegraphics[width=10cm]{img/clise_model.png}
\end{center}
}

\begin{frame}[fragile]
  \frametitle{Implementation}

  \begin{block}{Example/Input}
    \begin{lstlisting}
<link rel="stylesheet" type="text/css" href="css/jsxgraph.css" />
<script type="text/javascript" src="js/jsxgraphcore.js"></script>
...
<div id="jxgbox" class="jxgbox" style="width:500px; height:500px;"></div>
<script type="text/javascript">
  board = JXG.JSXGraph.initBoard('jxgbox', {boundingbox: [-2, 20, 20, -2], axis: true, grid: false, keepaspectratio: true});
  A = board.create('point', [8, 3]);
  B = board.create('point', [8, 8]);
  c1 = board.create('circle', [B, 4]);
  D = board.create('glider', [0, 0, c1], {name: 'D'});
  g = board.create('line', [A, D]);
  c2 = board.create('circle', [D, 3]);
  T = board.create('intersection', [c2,g,0], {name: 'T'});

  locus = board.create('locus', [T]);
</script>
    \end{lstlisting}
  \end{block}
\end{frame}

\frame{
  \frametitle{Implementation}

  \begin{block}{Example/Output}
\begin{center}
      \href{http://localhost/~michael/jxg/examples/adg/limacon.html}{\includegraphics[width=6.5cm]{img/locus.png}}
\end{center}
  \end{block}
}


\frame{
  \frametitle{Implementation}

  \begin{block}{Ready-to-use elements}
    \begin{itemize}
      \item Glider on circle and line
      \item Intersection points (circle/circle, circle/line, line/line)
      \item Midpoint
      \item Parallel line and point
      \item Perpendicular line and point
      \item Circumcircle and circumcenter
    \end{itemize}
  \end{block}
}


\frame{
  \frametitle{Implementation}

  \begin{block}{Easy to extend}
\begin{center}
      \href{http://localhost/~michael/jxg/examples/adg/limacon.html}{\includegraphics[width=8cm]{img/simsonsteiner.png}}
\end{center}
  \end{block}
}


\begin{frame}[fragile]
  \frametitle{Implementation}

  \begin{block}{}
    \begin{lstlisting}
<link rel="stylesheet" type="text/css" href="css/jsxgraph.css" />
<script type="text/javascript" src="js/jsxgraphcore.js"></script>
...
<div id="jxgbox" class="jxgbox" style="width:500px; height:500px;"></div>
<script type="text/javascript">
  board = JXG.JSXGraph.initBoard('jxgbox', {boundingbox:[-4, 6, 8, -4], axis: true, grid: false, keepaspectratio: true});
  A = board.create('point', [0, 0]);
  B = board.create('point', [6, 0]);
  C = board.create('point', [4, 4]);

  t1 = board.create('triangle', [A, B, C], {strokeWidth: '1px'});

  X = board.create('point', [4, 1.5], {name:"X"});

  L = board.create('perpendicularpoint', [X, t1.c]);
  M = board.create('perpendicularpoint', [X, t1.a]);
  N = board.create('perpendicularpoint', [X, t1.b]);

  t2 = board.create('triangle', [L, M, N], {strokeWidth: '1px'});
    \end{lstlisting}
  \end{block}
\end{frame}


\begin{frame}[fragile]
  \frametitle{Implementation}

  \begin{block}{}
    \begin{lstlisting}
  ...

  X.ancestors[L.id] = L;
  X.ancestors[M.id] = M;
  X.ancestors[N.id] = N;
  X.ancestors[A.id] = A;
  X.ancestors[B.id] = B;
  X.ancestors[C.id] = C;

  X.generatePolynomial = function () {
    var as16 = getTriangleArea(L, M, N),
    as = '((('+M.symbolic.x+')-('+N.symbolic.x+'))^2+(('+M.symbolic.y+')-('+N.symbolic.y+'))^2)',
    bs = '((('+L.symbolic.x+')-('+N.symbolic.x+'))^2+(('+L.symbolic.y+')-('+N.symbolic.y+'))^2)',
    cs = '((('+M.symbolic.x+')-('+L.symbolic.x+'))^2+(('+M.symbolic.y+')-('+L.symbolic.y+'))^2)',

    return ['4*'+as+'*'+cs+'-('+as+'+'+cs+'-'+bs+')*('+as+'+'+cs+'-'+bs+')-('+as16+')'];
  };

  locus = board.create('locus', [X], {strokeColor: 'red'});
</script>
    \end{lstlisting}
  \end{block}
\end{frame}


\frame{
  \frametitle{Implementation}

  \begin{block}{Re-using locus data: Discovered loci can be}
    \begin{itemize}
      \item intersected with circles, lines, other curves, ...
      \item used as a base object for gliding points
      \item used for the discovery of other loci
    \end{itemize}
  \end{block}
}

\begin{frame}[fragile]
  \frametitle{Implementation}
\begin{columns}
        \column{.40\textwidth}
	    \href{http://localhost/~michael/jxg/examples/adg/limacon.html}{\includegraphics[width=4.5cm]{img/limacon-ext.png}}
        \column{.60\textwidth}
    \begin{lstlisting}
C = board.create('glider', [loc]);

c2 = board.create('circle', [C, 3]);
E = board.create('intersection', [c1, c2, 0]);
F = board.create('midpoint', [C, E]);
    \end{lstlisting}
\end{columns}
\end{frame}


\frame{
  \frametitle{Implementation}

\begin{center}
      \href{http://localhost/~michael/jxg/examples/adg/limacon.html}{\includegraphics[width=7cm]{img/limacon-extloc.png}}
\end{center}
}



\section{Optimizations}

\frame{
  \frametitle{}

  \begin{block}{Optimization}
  \end{block}
}


\begin{frame}[fragile]
  \frametitle{Optimization}
  \begin{block}{Idea: Divide and conquer}
    \begin{columns}
      \column{.50\textwidth}
	    \href{http://localhost/~michael/jxg/examples/adg/limacon.html}{\includegraphics[width=5cm]{img/limacon-extconstr.png}}
      \column{.50\textwidth}
          \begin{tikzpicture}[scale=0.5,node distance=1cm,>=stealth',bend angle=45,auto]

  \tikzstyle{place}=[circle,thick,draw=black!75,fill=black!20,minimum size=6mm]
  \tikzstyle{red place}=[place,draw=red!75,fill=red!20]
  \tikzstyle{transition}=[rectangle,thick,draw=black!75,
  			  fill=black!20,minimum size=6mm]

  \tikzstyle{every label}=[black]

  \begin{scope}
    \node [place] (E) [label=right:2, scale=0.75] {E};
    \node [place] (T) [below of=E, xshift=-45, label=left:2,scale=0.75] {T}
      edge [pre] (E);
    \node [place] (D) [below of=T, xshift=-30, label=left:1,scale=0.75] {D}
      edge [pre] (T);
    \node [transition] (A) [below of=T, xshift=30,scale=0.75] {A}
      edge [pre] (T);
      
    \node [place] (C) [below of=E, xshift=45, label=right:2,scale=0.75]  {C}
      edge [pre] (E)
      edge [post] (T);
    \node [transition] (B) [below of=C,scale=0.75] {B}
      edge [pre] (C)
      edge [pre,bend left] (D);
  \end{scope}

          \end{tikzpicture}
    \end{columns}
  \end{block}
\end{frame}

\frame{
  \frametitle{Optimization}
  
  \begin{block}{Transformations}
    \begin{itemize}
      \item Translate the construction moving one point to $(0, 0)$
      \item Rotate the construction moving another point onto the x-axis
      \item After the Gr\"obner basis is calculated, the result is retransformed
      \item User can choose the two points or
      \item JSXGraph chooses two points (but sometimes not the best suited ones)
    \end{itemize}  
  \end{block}

}

\section{Examples}

\frame{
  \frametitle{}

  \begin{block}{Examples}
  \end{block}
}


\frame{
  \frametitle{Last slide}

  \begin{block}{Thank You}
    \begin{itemize}
      \item \href{http://jsxgraph.org/}{http://jsxgraph.org/}
      \item \href{http://jsxgraph.uni-pdflatebayreuth.de/wiki/}{http://jsxgraph.uni-bayreuth.de/wiki/}
    \end{itemize}
  \end{block}
}

\end{document}
